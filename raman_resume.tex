%-------------------------
% Resume in Latex
% Author : Raman Shinde
% License : MIT
%------------------------

\documentclass[letterpaper,11pt]{article}

\usepackage{latexsym}
\usepackage[empty]{fullpage}
\usepackage{titlesec}
\usepackage{marvosym}
\usepackage[usenames,dvipsnames]{color}
\usepackage{verbatim}
\usepackage{enumitem}
\usepackage[hidelinks]{hyperref}
\usepackage{fancyhdr}
\usepackage[english]{babel}

\pagestyle{fancy}
\fancyhf{} % clear all header and footer fields
\fancyfoot{}
\renewcommand{\headrulewidth}{0pt}
\renewcommand{\footrulewidth}{0pt}

% Adjust margins
\addtolength{\oddsidemargin}{-0.5in}
\addtolength{\evensidemargin}{-0.5in}
\addtolength{\textwidth}{1in}
\addtolength{\topmargin}{-.5in}
\addtolength{\textheight}{1.0in}

\urlstyle{same}

\raggedbottom
\raggedright
\setlength{\tabcolsep}{0in}

% Sections formatting
\titleformat{\section}{
  \vspace{-4pt}\scshape\raggedright\large
}{}{0em}{}[\color{black}\titlerule \vspace{-5pt}]

%-------------------------
% Custom commands
\newcommand{\resumeItem}[2]{
  \item\small{
    \textbf{#1}{: #2 \vspace{-2pt}}
  }
}

\newcommand{\resumeSubheading}[4]{
  \vspace{-1pt}\item
    \begin{tabular*}{0.97\textwidth}[t]{l@{\extracolsep{\fill}}r}
      \textbf{#1} & #2 \\
      \textit{\small#3} & \textit{\small #4} \\
    \end{tabular*}\vspace{-5pt}
}

\newcommand{\resumeSubhead}[6]{
  \vspace{-1pt}\item
    \begin{tabular*}{0.97\textwidth}[t]{l@{\extracolsep{\fill}}r}
      \textbf{#1} & #2 \\
      \textit{\small#3} & \textit{\small #4} \\
      \textit{\small#5} & \textit{\small #6} \\
    \end{tabular*}\vspace{-5pt}
}


\newcommand{\resumeCase}[1]{
  \vspace{-1pt}\item
    \begin{tabular*}{0.97\textwidth}[t]{l@{\extracolsep{\fill}}r}
      \textbf{#1}
    \end{tabular*}\vspace{-5pt}
}


\newcommand{\resumeBlog}[2]{
  \vspace{-1pt}\item
    \begin{tabular*}{0.97\textwidth}[t]{l@{\extracolsep{\fill}}r}
      \textbf{#1}\\
      \textsl{\small#2}\\
    \end{tabular*}\vspace{-5pt}
}




\newcommand{\resumeSubItem}[2]{\resumeItem{#1}{#2}\vspace{-4pt}}

\newcommand{\resumePoint}[1]{\item\small{{#1}\vspace{-4pt}}}

\renewcommand{\labelitemii}{$\circ$}

\newcommand{\resumeSubHeadingListStart}{\begin{itemize}[leftmargin=*]}
\newcommand{\resumeSubHeadingListEnd}{\end{itemize}}
\newcommand{\resumeItemListStart}{\begin{itemize}}
\newcommand{\resumeItemListEnd}{\end{itemize}\vspace{-5pt}}

%-------------------------------------------
%%%%%%  CV STARTS HERE  %%%%%%%%%%%%%%%%%%%%%%%%%%%%


\begin{document}

%----------HEADING-----------------
\begin{tabular*}{\textwidth}{l@{\extracolsep{\fill}}r}
  \textbf{\href{https://www.linkedin.com/in/raman-shinde-562950b2}{\Large Raman Shinde}} & Email : \href{mailto:raman.shinde15@gmail.com}{raman.shinde15@gmail.com}\\
  \href{https://github.com/Raman-Raje}{https://github.com/Raman-Raje} & Mobile : +91 9595161238 \\
\end{tabular*}


%--------CASE STUDIES------------
\section{Projects}
\resumeSubHeadingListStart

 \resumeCase
    {Deep Learning}
      \resumeItemListStart
	\resumeItem{\href{https://github.com/Raman-Raje/Neural-Machine-Translation-using-Attention-mechanism}{Neural Machine Translation using Attention mechanism (NLP)}}
	{Task is to implement Machine Translator.Attention was used to deal with longer sequences. Data cleaning and output labels were padded with start and end tokens before feeding to n/w.}
	
	\resumeItem{\href{https://github.com/Raman-Raje/Machine-Reading-Comprehension-Neural-Question-Answer-} {Nueral Question Answering(NLP + Attention +  Machine Reading Coprehension)}}
		{Objective is to find correct answer for given question and context pair. Implemented Standford Attentive Reader.SQUAD v1 dataset was used for this project. Various binary and NLP features were used to get the best results. Compared the final results with fine tuned BERT model.}    
	
	 \resumeItem{\href{https://github.com/Raman-Raje/ImageCaptioning} {Image Captioning with Flickr 8k dataset. (CNN+RNN+Transfer Learning)}}
		{Task is to generate caption for an image.Pre-trained Inception n/w on imagenet dataset was used in combination with RNN to generate caption for the given image. One-to-many RNN is used to train the n/w. For each image 5 standard caption are provided. The data was converted into sequence before feeding the n/w.}

    \resumeItemListEnd

    \resumeCase
      {Machine Learning}
      \resumeItemListStart
        
		\resumeItem{\href{https://github.com/Raman-Raje/Netflix-Movie-Recommendation-}{Netflix Movie Recommendation System (Collaborative based recommendation)}}
          {Objective was for the given movie and user predict the rating given by him/her to the movie.The dataset was obtained from kaggle. Matrix factorization was used to get similarity matrices. Tried and tested various ML models to get minimum Root Mean Square.}

		\resumeItem{\href{https://github.com/Raman-Raje/Stack-Overflow-Tag-Prediction} {Stack Overflow Tag Prediction}}
		{Objective is to predict as many as tags possible with high Precision and Recall.The dataset was obtained from kaggle. The given problem is \textbf{multi-label classification problem}. The dataset contains features such as Id, Title, Body and Tags. Data preprocessing and cleaning was done to remove html tags and hyperlinks. Micro-Averaged F1-Score was used as performance metric as mentioned on Kaggle. }

      \resumeItemListEnd
	
 
 \resumeSubHeadingListEnd

%-----------EXPERIENCE-----------------
\section{Experience}
  \resumeSubHeadingListStart

    \resumeSubheading
      {Siemens R\&D}{Pune, India}
      {Product Development Engineer}{Dec. 2018 - Present}
      \resumeItemListStart
        \resumeItem{Automation Designer(Python/C++)}
          {Working as a part of sequence designer team. Working on NX as a product developer for Siemens.}
      \resumeItemListEnd

    \resumeSubheading
      {TCS}{Pune, India}
      {Software Developer}{Dec 2015 - Nov 2018}
      \resumeItemListStart
        \resumeItem{Application Developer(Python)}
          {Developed an application for client NCRA for Monitoring and Controlling of Giant Metrewave Radio Telescope. Coding for I/O operation. Monitoring H/W devices. Debugging the issue and modifying the GUI as per requirements}
        \resumeItem{Production Management}
          {Worked in QAPM for client Morgan Stanley.In QAPM team, I had worked in ED\&S (Enterprise data and services) team which deals with different applications like CRD, SRD, TS, and PM}
      \resumeItemListEnd
	  
  \resumeSubHeadingListEnd

%%--------CERTIFICATION------------
%
%\section{Courses/Internship}
% \resumeSubHeadingListStart
% 	\resumePoint{Applied Machine Learning course at Applied AI. ( Jan 2018 to May 2019)}
% 	\resumePoint{Internship at IARE, Aurangabad on Industrial automation. (May 2014 - Jun 2014)}
% 	
% \resumeSubHeadingListEnd
   
%-----------WRITTINGS----------------------
\section{Blogs}
	\resumeSubHeadingListStart
 	\resumePoint{Understanding sequential/Time-Series data for LSTM}
 	\resumePoint{Back Propagation through LSTM: A differential approach}
 	\resumePoint{Image Captioning With Flickr8k Dataset \& BLEU}
 	\resumePoint{Neural Question And Answering Using SQAD Dataset And Attention..!!!}
 	\resumeSubHeadingListEnd
   
%--------ACADEMICS------------

\section{Education}
  \resumeSubHeadingListStart
    \resumePoint{\textbf{B.Tech} in Electronics and Telecommunication from Shri Guru Gobind Singhji Institute of Engineering and Technology, Nanded with \textbf{CGPA 7.7} (2011 - 2015)}
    \resumePoint{Class Xll (HSC), form Maharashtra State Board of Education with \textbf{83.33\%} (2009 - 2011)}
     \resumePoint{Class X (SSC), form Maharashtra State Board of Education with \textbf{90.92\%}  (2008 - 2009)}
  \resumeSubHeadingListEnd  

 %--------TECHNICAL SKILLS------------
\section{Technical Skills}
  \resumeSubHeadingListStart
    \resumeSubItem{Languages}{Python, C++, C}
    \resumeSubItem{Data Analysis}{Pandas, Numpy, Matplotlib, Seaborn}
    \resumeSubItem{ML /DL Toolkit}{Keras, Sklearn, scikit-multilearn , tensorflow}
  \resumeSubHeadingListEnd

%-------------------------------------------
\end{document}
